\documentclass[a4paper,12pt]{article}

\usepackage[left=2.5cm, right=2.5cm, top=3cm, bottom=3cm]{geometry}
\usepackage{amsmath, amsthm, amssymb}
\usepackage[spanish]{babel}
\usepackage{cite}
\usepackage{url}

\bibliographystyle{plain}

\begin{document}
\title{Reporte del Moogle!}
\author{Luis Alejandro Arteaga Morales}
\date{Julio, 2023}
\maketitle

\begin{abstract}
    Moogle! es un motor de búsqueda sencillo que corre desde el navegador y permite al usuario a realizar búsquedas de documentos en formato .txt que contengan un input en una base de datos local pequeña.
    Está implementado en C\# y utiliza como criterio de búsqueda el TFIDF sobre los documentos y similitud de cosenos.
    \\
    \newline
    {\bf Keywords:} motor de búsqueda, navegador, .txt, local, C\#, TFIDF, similitud de cosenos.
\end{abstract}


\section{Introducción}\label{sec:intro}

En el presente artículo se tocarán las ideas principales que hay tras la implementación de Moogle! así como un tour por sus funcionalidades básicas.

\section{Ideas principales}\label{sec:core-ideas}

La implementación del Motor de Búsqueda está basada en el modelo vectorial, tomando cada consulta(query) y 
cada texto como un vector de palabras de $\mathbb{R}^{n}$, representados con el valor
de TF-IDF de cada palabra. Luego se emplea la similitud de cosenos para ver qué tan cerca de
ser iguales están dos vectores. De esta manera se tiene una noción de qué documentos son los
más similares a lo que buscamos.

\section{Funcionalidades básicas}\label{sec:funct}

\subsection{Búsqueda simple}\label{sub:simplesearch}
Se le mostrará una lista de documentos que contengan las palabras que aparecen en su búsqueda. Si escribe 
una consulta vacía el buscador se lo informará. Se mostrarán primero los documentos que contienen palabras 
más raras (aparecen en la consulta pero aparecen en menos documentos). Las palabras que son demasiado comunes
(aparecen en todos los documentos) no serán relevantes y por tanto ignoradas por la búsqueda

\subsection{Operador !}\label{sub:ignore}
Los documentos que contengan las palabras que tengan un signo ! delante en la consulta, no serán sugeridos.

\subsection{Operador \textasciicircum}\label{sub:force}
Solo podrán ser sugeridos documentos que contengan las palabras que tengan un signo \textasciicircum delante en la consulta.

\subsection{Operador *}\label{sub:increment}
Cualquier cantidad de * delante de una palabra aumenta la relevancia de esta para la búsqueda

\subsection{Operador \textasciitilde}\label{sub:near}
Para dos o más palabras unidas por \textasciitilde, entre más cerca estén esas palabras en un documento más relevante será este.

\subsection{Sugerencias}\label{sub:suggest}
Si no se encuentra ningún resultado para su búsqueda se mostrará una sugerencia de lo que debería escribir en su 
consulta para tener resultados. Esta sugerencia está basada en palabras ya contenidas en los textos que sean lo 
más parecidas a la consulta inicial.

\section{Estructura del proyecto}\label{sec:struct}

\subsection{Estructura general}\label{sub:general-struct}
El proyecto está dividido en dos partes principales: MoogleServer, que contiene el
servidor e implementación del Frontend del motor y MoogleEngine, que contiene la lógica que
hay detrás del motor. Nos centraremos en describir esta última en el reporte.
Durante el análisis algunos detalles de implementación serán omitidos. De lo contrario el
documento sería demasiado extenso. Para ver estos se pueden revisar los comentarios en el
código fuente del programa.
\\
\newline
MoogleEngine está compuesto de las siguientes clases, que iremos desglosando en las
secciones que siguen:

\begin{itemize}
    \item Document
    \item DocumentCatcher
    \item SearchItem
    \item SearchResult
    \item TFIDFAnalyzer
    \item StringMethods
    \item Moogle
    \item Trie (para futuras mejoras del buscador)
    \item Matrix (por propósitos de evaluación de la asignatura)
\end{itemize}

\subsection{Document}\label{sub:doc}
Representa objetos de tipo documento, con las propiedades que se utilizan
de ellos para cada una de las consultas.
\begin{itemize}
    \item public string Title(título del documento)
    \item public string Text (texto del documento)
    \item public string LowerizedText (texto en minúsculas, para solo preocuparnos por la semántica de los
    datos y no de la capitalización, esta propiedad es usada en la construcción de snippets
    para la búsqueda en la clase Moogle)
    \item public string[] Words (contiene un arreglo con todas las palabras del texto separadas, esta separación
    está dada por los caracteres que no constituyen letras o números, ej: signos de
    puntuación)
    \item public Dictionary $<$ string, int $>$ WordFrequency (contiene cuantas veces aparece cada palabra en el texto)
    \item public Dictionary $<$ string, float $>$ TF (representa el vector con los valores de TF para cada palabra, en el desglose de la
    clase TFIDFAnalyzer veremos que significa el TF y el IDF en cada documento)
    \item public Dictionary $<$ string, float $>$ TFIDF (vector de TF * IDF de cada palabra, estos son los vectores que utilizaremos en
    nuestra búsqueda)
    \item public Dictionary $<$ string, List$<$int$>$ $>$ WordPos (posiciones de cada palabra en el documento para la implementación del
    operador \textasciitilde en la clase Moogle)
\end{itemize}

\subsection{DocumentCatcher}\label{sub:doc-catcher}
Esta clase estática contiene los métodos y constantes necesarios para procesar los
archivos y convertirlos a nuestra clase Document.
\\
\newline Métodos:
\begin{itemize}
    \item public static string IgnoreASCII() genera todos los caracteres que ignoraremos al procesar los textos y
    actuarán como separadores de las palabras. Internamente recorre todos los caracteres
    ASCII y considera todos los que no son letras o números.
    \item public static string IgnoreASCIIQuery() este método funciona como el anterior pero no elimina los
    caracteres especiales para los operadores de búsqueda, véase: ! * \textasciicircum \textasciitilde
    \item public static bool IsAlphanumeric(char c) comprueba si un carácter es letra o número (esta función realiza la
    misma función que char.IsLetterOrDigit y sólo permanece en el código por motivos de
    seguridad, en versiones posteriores puede ser refactorizada y eliminada).
    \item public static bool IsOperator(char c) comprueba si un caracter es uno de los cuatro operadores especiales (es una
    función de utilidad empleada en varios lugares del código).
    \item public static bool InvalidWord(string word) comprueba si una palabra es inválida o no basándose en si contiene letras
    o números y no símbolos extraños.
    \item public static Document[] ReadDocumentsFromFolder(string folder) acepta un directorio en forma de string como parámetro y
    devuelve un arreglo de Document, construido a partir de los .txt contenidos en este
    directorio. Esta es la función que utilizamos para construir los documentos sobre los que
    realizaremos la query.
\end{itemize}

Propiedades:
\begin{itemize}
    \item public static char[] Delims contiene todos los caracteres que actúan como separadores para las palabras de
    los documentos (es construido a partir de IgnoreASCII() )
    \item public static char[] DelimsQuery contiene todos los caracteres que actúan como separadores para la query
    (es construido a partir de IgnoreASCIIQuery() )
\end{itemize}

\subsection{SearchItem}\label{sub:search-item}
Representa los elementos devueltos en la query, e implementa de la clase IComparable
para poder ordenar usando Array.Sort()
Propiedades:
\begin{itemize}
    \item public string Title (título del documento devuelto)
    \item public string Snippet (fragmento de texto del documento que contiene al menos una palabra de la
    query)
    \item public float Score (puntuación del documento empleada para poderlos ordenar de mayor a menor
    relevancia)
\end{itemize}

Nota: El método CompareTo permite comparar dos SearchItem en base a sus puntuaciones.

\subsection{SearchResult}\label{sub:search-result}
Contiene los resultados de la búsqueda y una sugerencia de cómo podríamos obtener
mejores resultados de la búsqueda (esta será la misma que la query si encontramos resultados
relevantes)
\begin{itemize}
    \item private SearchItem[] items (arreglo de SearchItem que contiene los resultados de la búsqueda)
    \item public string Suggestion (string que contiene una nueva query con la que podríamos mejorar los
    resultados)
\end{itemize}

\subsection{TFIDFAnalyzer}\label{sub:tfidf}
Esta clase estática contiene los métodos necesarios para poder construir los vectores de
TF-IDF necesarios para procesar las consultas(queries).
Un vector de TF-IDF está formado digamos por $N$ componentes(donde cada una
representa a una palabra), que son números reales. Estos números reales que representaremos
dan una cierta noción de que tan raras e importantes o no son cada una de las palabras que
estamos procesando en nuestras queries y documentos, y están dadas por las fórmulas siguientes:

\begin{equation}\label{eq:tf-formula}
    TF(Palabra, Documento) = \frac{cntPalabra}{cntDoc}
\end{equation}

Donde $cntPalabra$ es la cantidad de ocurrencias de la palabra en el documento y
$cntDoc$ es la cantidad de palabras en el documento

\begin{equation}\label{eq:idf-formula}
    IDF(Palabra) = \ln{\frac{Docs}{PalabraDocs}}
\end{equation}

Donde $Docs$ es la cantidad de documentos y $PalabraDocs$ es la cantidad de documentos donde aparece la palabra

Luego el TF-IDF es el producto de estas dos medidas. Notemos que si una palabra
aparece en todos los documentos su IDF será $\ln{1} = 0$, y el producto del TF-IDF será 0
tambíen, por lo que la palabra no será relevante.

Este tipo de palabras no relevantes son llamadas StopWords y no nos interesan en nuestra
búsqueda.

Propiedades:
\begin{itemize}
    \item public static void CalculateDocumentsWithTerm() recorre todas las palabras del vocabulario y
    determina cuantos documentos contienen a cada una, esta información es
    almacenada en Dictionary $<$ string,int $>$ DocumentsWithTerm en Moogle.cs
    \item public static void CalculateIDFVector() calcula el vector de IDF para todos los documentos y lo
    almacena en IDFVector en Moogle.cs
    \item public static void CalculateTFVector(Document doc) calcula el vector de TF para cada documento
    \item public static void CalculateTFIDFVector(Document doc) una vez calculado el vector de IDF y los vectores de TF
    construye el vector de TFIDF para cada documento.
    \item public static float Norm(Dictionary $<$ string, float $>$ vec) recibe un Dictionary $<$ string,float $>$ vec, que no es más que un vector de
    TF-IDF y calcula su norma, que es el valor por el cual tenemos que dividir cada
    valor en el vector para tener un vector unitario, de cierta manera esto es parecido a
    la Fórmula de Pitágoras pero generalizada a $N$-dimensiones, ya que nos da algo
    así como el tamaño del vector. Este método será útil a continuación cuando
    veamos ComputeRelevance(). Su valor es la raíz cuadrada de la sumatoria de los
    cuadrados de cada componente.
    \item public static float ComputeRelevance(Document query, Document doc) permite determinar que tan relevante es un documento. Para
    esto como ya tenemos representados los vectores de TF-IDF por decir así como
    vectores del espacio $\mathbb{R}^{N}$ entonces queremos ver que tan cerca de ser colineales
    están esos vectores, porque esto nos dice que tan cerca de ser iguales (de contener
    lo mismo) están, para ello usaremos la siguiente fórmula:
    \begin{equation}\label{eq:cosine-similarity}
        Relevance = \frac{v \times w}{\|v\| \times \|w\|}
    \end{equation}
    para el cálculo del numerador recorreremos cada componente del vector de la
    query verificando que también esté en el documento y haremos producto de
    vectores con estos. Luego para el denominador usaremos la función Norm().
    En la implementación se tiene cuidado con los denominadores nulos, para no
    causar indefiniciones y errores en tiempo de ejecución.
\end{itemize}

\subsection{StringMethods}\label{sub:string}
Esta clase estática contiene los métodos necesarios para poder construir la sugerencia en
Moogle.cs cuando no encontramos resultados

\begin{itemize}
    \item public static int EditDistance(string s1, string s2) devuelve la distancia entre dos strings que recibe como
    parámetro, es decir la mínima cantidad de caracteres que hay que insertar,
    remover o cambiar para que los dos strings sean iguales
    \item public static int LongestCommonSubsequence(string s1, string s2) devuelve la subsecuencia común más larga entre
    dos strings que recibe como parámetro.
\end{itemize}

Ambos métodos son calculados usando Programación Dinámica para lograr mayor eficiencia.

\subsection{Moogle}\label{sub:moogle}
Este es el núcleo de nuestro proyecto y donde procesamos las consultas a mostrar.
Tiene un constructor estático el cual es accedido por Index.razor en MoogleServer para
precalcular la información necesaria para las queries.

\begin{itemize}
    \item static void Preproccess() es llamado por static Moogle() para precalcular la información de los
    documentos, construye los documentos y el cuerpo de vectores de TFIDF(corpus)
    utilizando los métodos de TFIDFAnalyzer, Document y DocumentCatcher.
    \item static void ResetGlobalVariables() es llamado cada vez que hacemos una nueva query para limpiar la
    información de la query anterior y poder calcular la nueva sin errores.
    \item static void CalculateNewRelevance() incrementa la relevancia de las palabras en el vector de TF-IDF
    de la query multiplicando por 2.0f por cada asterisco encontrado en el prefijo de la
    palabra.
    \item static void UpdateNewRelevance() actualiza la nueva relevancia en el vector de la query
    \item static void FindSearchResults(List $<$ SearchItem $>$ results) inserta a la lista de resultados los documentos más relevantes para la
    búsqueda. Considera los 4 operadores especiales. Recorre todas las palabras de la query y
    analiza todos los documentos, solo considerando aquellos que tienen palabras
    obligatorias (de haberlas) y que no tienen palabras prohibidas, sugiriendo los que mayor
    score tienen, y de todos estos en caso de existir el operador \textasciitilde en la query se incrementa la 
    relevancia según las palabras cercanas. Luego se construye el snippet del documento y se
    añade a la lista. Se usan expresiones regulares (RegEx) para construir el snippet, osea
    para encontrar el fragmento de texto que contiene palabras de la query. En caso de no
    poder construir el snippet por algún símbolo extraño haberlo impedido, entonces no se
    sugiere.
    \item static string BuildSuggestion() devuelve un string que es la sugerencia de lo que debería ser escrito en
    el input para obtener un resultado en la búsqueda. Recorre todas las palabras de la query
    y las sustituye por la palabra con menor EditDistance, de haber varias, escoge la de
    menor LongestCommonSubsequence.
    \item static string BuildSnippet(string[] words, string originalText, string lowerizedText) construye el snippet a partir de un documento dado (recibe como
    parámetros las palabras del documento, el texto original, y el texto en minúscula) y
    recorre las palabras de la query, cuando encuentra una de ellas intenta mostrar 20
    caracteres a la izquierda y 200 a la derecha y devuelve este intervalo en el texto original.
    \item static List $<$ Tuple $<$ string, string $>$ $>$ FindWordsToBeNear() construye una lista que relaciona que palabras tienen que estar
    cerca de otras para el operador \textasciitilde. Devuelve un objeto de tipo List$<$Tuple$<$int,int$>$$>$.
    \item public static SearchResult Query(string query) recibe un string query que es obtenido desde Index.razor y encuentra todos los
    documentos relevantes ordenados utilizando los métodos anteriores. Los devuelve como
    un SearchResult.
\end{itemize}

\subsection{Trie}\label{sub:trie}
Esta clase se encuentra en el programa para poder ser usada en funcionalidades futuras.
Ej: Sugerencia de palabras en tiempo real, búsqueda de palabras con el mismo prefijo (esta
funcionalidad ya se encontraba en el programa pero tuvo que ser removida porque es demasiado
costosa en memoria)

\subsection{Matrix}\label{sub:matrix}
Esta clase está en el proyecto solo para evaluar los elementos básicos de álgebra (multiplicación de matrices y demás)

\section{Conclusiones}\label{sec:concl}

Ante cualquier duda visitar el repositorio público del proyecto en \url{https://github.com/Sekai02/moogle-1}

\newpage

\tableofcontents

\newpage
\bibliography{bibliography}
\end{document}